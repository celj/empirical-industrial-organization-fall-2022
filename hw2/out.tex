% see http://info.semprag.org/basics for a full description of this template
\documentclass[cm,linguex]{glossa}

% possible options:
% [times] for Times font (default if no option is chosen)
% [cm] for Computer Modern font
% [lucida] for Lucida font (not freely available)
% [brill] open type font, freely downloadable for non-commercial use from http://www.brill.com/about/brill-fonts; requires xetex
% [charis] for CharisSIL font, freely downloadable from http://software.sil.org/charis/
% for the Brill an CharisSIL fonts, you have to use the XeLatex typesetting engine (not pdfLatex)
% for headings, tables, captions, etc., Fira Sans is used: https://www.fontsquirrel.com/fonts/fira-sans
% [biblatex] for using biblatex (the default is natbib, do not load the natbib package in this file, it is loaded automatically via the document class glossa.cls)
% [linguex] loads the linguex example package
% !! a note on the use of linguex: in glossed examples, the third line of the example (the translation) needs to be prefixed with \glt. This is to allow a first line with the name of the language and the source of the example. See example (2) in the text for an illustration.
% !! a note on the use of bibtex: for PhD dissertations to typeset correctly in the references list, the Address field needs to contain the city (for US cities in the format "Santa Cruz, CA")

%\addbibresource{sample.bib}
% the above line is for use with biblatex
% replace this by the name of your bib-file (extension .bib is required)
% comment out if you use natbib/bibtex

\let\B\relax %to resolve a conflict in the definition of these commands between xyling and xunicode (the latter called by fontspec, called by charis)
\let\T\relax
\usepackage{xyling} %for trees; the use of xyling with the CharisSIL font produces poor results in the branches. This problem does not arise with the packages qtree or forest.
\usepackage[linguistics]{forest} %for nice trees!
\usepackage{longtable}

\title[Empirical Industrial Organization]{Solving and Estimating\\
an Incomplete Information Entry Game}
% Optional short title inside square brackets, for the running headers.

% \author[Paul \& Vanden Wyngaerd]% short form of the author names for the running header. If no short author is given, no authors print in the headers.
% {%as many authors as you like, each separated by \AND.
%   \spauthor{Waltraud Paul\\
%   \institute{CNRS, CRLAO}\\
%   \small{105, Bd. Raspail, 75005 Paris\\
%   waltraud.paul@ehess.fr}
%   }
%   \AND
%   \spauthor{Guido Vanden Wyngaerd \\
%   \institute{KU Leuven}\\
%   \small{Warmoesberg 26, 1000 Brussel\\
%   guido.vandenwyngaerd@kuleuven.be}
%   }%
% }

\author[Lezama]{
    \spauthor{Carlos Enrique Lezama Jacinto\\
  \institute{\hfill\break
Instituto Tecnológico\\
Autónomo de México}\\
  \small{\hfill\break
clezamaj@itam.mx}
  }%
  }

\usepackage{natbib}


% tightlist command for lists without linebreak
\providecommand{\tightlist}{%
  \setlength{\itemsep}{0pt}\setlength{\parskip}{0pt}}

% From pandoc table feature
\usepackage{longtable,booktabs,array}
\usepackage{calc} % for calculating minipage widths
% Correct order of tables after \paragraph or \subparagraph
\usepackage{etoolbox}
\makeatletter
\patchcmd\longtable{\par}{\if@noskipsec\mbox{}\fi\par}{}{}
\makeatother
% Allow footnotes in longtable head/foot
\IfFileExists{footnotehyper.sty}{\usepackage{footnotehyper}}{\usepackage{footnote}}
\makesavenoteenv{longtable}




\usepackage{multirow, array}
\setlength{\parskip}{\baselineskip}
\newcommand{\ev}{\text{E}}
\newcommand{\iid}{\text{i.i.d.}}
\newcommand{\dnorm}{\mathcal{N}}
\newcommand{\dunif}{\mathcal{U}}

\begin{document}


\sffamily
\maketitle



\rmfamily

%  Body of the article
\hypertarget{setup}{%
\section{Setup}\label{setup}}

Suppose there are two firms, \(A\) and \(B\), that simultaneously decide
whether to enter in the market \(m \in \{1, \dots, M\}\) or not. If firm
\(i \in \{A, B\}\) decides to enter, its profits are given by

\[
\Pi_{i, m} = X_{i, m} \beta - D_{j, m} \alpha + \nu_m + u_{i, m}, \quad \forall\ i, j \in \{A, B\},\ i \neq j
\]

where \(X_{i, m}\) is an observable characteristic of firm \(i\) that
boosts profits in market \(m\), \(D_{j, m} \in \{0, 1\}\) is \(j\)'s
decision of entering the market, \(\nu_m\) is a market fixed effect, and
\(u_{i, m}\) captures unobserved idiosyncratic shocks to profits.

On the other hand, if firm \(i\) decides not to enter market \(m\), its
profits are zero.

Notice that \(u_{i, m}\) is observed by \(i\) but not by its opponent
\(j\). Consequently, the firms are playing an incomplete information
entry game described below.

\begin{center}
\begin{tabular}{cc|c|c|}
    & \multicolumn{1}{c}{} & \multicolumn{2}{c}{Firm $B$}\\
                            & \multicolumn{1}{c}{} & \multicolumn{1}{c}{Enter}                             & \multicolumn{1}{c}{Do not enter}                      \\\cline{3-4}
    \multirow{2}*{Firm $A$} & Enter                & $\left(\Pi_{A ,m}, \Pi_{B, m}\right)$                 & $\left( X_{A, m} \beta + \nu_m + u_{A, m}, 0 \right)$ \\\cline{3-4}
                            & Do not enter         & $\left( 0, X_{B, m} \beta + \nu_m + u_{B, m} \right)$ & $\left(0, 0\right)$                                   \\\cline{3-4}
\end{tabular}
\end{center}

Thus, the expected profits of firm \(A\) if it decides to enter market
\(m\) are

\begin{align*}
    \ev\left(\Pi_{A, m}\right) & = \left( D_{A, m} \right) \left( p(D_{B, m} = 1) \left( X_{A, m} \beta - \alpha + \nu_m + u_{A, m} \right) \right. \\ &+ \left. \left( 1 - p(D_{B, m} = 1)\right) \left( X_{A, m} \beta - \alpha + u_{A, m} \right) \right) \\
    &+ \left( 1 - D_{A, m} \right) \cdot 0
\end{align*}

where, \(\forall\ i, j \in \{A, B\}\), \(i \neq j\), the condition for
firm \(i\) to enter the market in terms of its expected profits is
defined as follows:

\begin{align*}
    D_{i, m} & = \mathbf{1} \left( p(D_{j, m} = 1) \left( X_{i, m} \beta - \alpha + \nu_m + u_{i, m} \right) \right. \\ &+ \left. \left( 1 - p(D_{j, m} = 1) \right) \left( X_{i, m} \beta + \nu_m + u_{i, m} \right) > 0 \right) .
\end{align*}

Namely, the probability that firm \(A\) enters market \(m\) is

\begin{align*}
    p \left(D_{A, m} = 1\right) & = p \left( X_{A, m} \beta - \alpha p \left( D_{B, m} = 1 \right) + \nu_m + u_{A, m} > 0 \right)              \\
                                    & = p \left( u_{A, m} > - \left( X_{A, m} \beta - \alpha p \left( D_{B, m} = 1 \right) + \nu_m \right) \right) \\
                                    & = \Phi \left( X_{A, m} \beta - \alpha p \left( D_{B, m} = 1 \right) + \nu_m \right)                              
\end{align*}

where \(\Phi = \dnorm (0, 1)\). Recall that we assume
\(u_{i, m} \sim \dnorm(0, 1)\), \(\forall\ i, j \in \{A, B\}\),
\(i \neq j\).

Similarly,

\[
p \left( D_{B, m} = 1 \right) = \Phi \left( X_{B, m} \beta - \alpha p \left( D_{A, m} = 1 \right) + \nu_m \right) .
\]

Henceforth, let \(p_{i, m} = p(D_{i, m} = 1)\).

\newpage

\hypertarget{simulations}{%
\section{Simulations}\label{simulations}}

We propose the following parametrization to simulate our model:

\begin{align*}
&\alpha =  2 && X_{A,m} \sim \dunif(0, 1) \\
&\beta = 0.2 && X_{B, m} \sim \dunif(0.1, 1.4) \\
&\nu_m = 0.9 \times \mathbf{1} \{ m \leq 100 \} && u_{A, m}, u_{B, m} \overset{\iid}{\sim} \dnorm(0, 1)
\end{align*}

With this setup, we obtain the following entry probabilities for markets
1, 101, 201, 301, and 401.

\begin{longtable}[]{@{}ccc@{}}
\toprule()
\(m\) & \(p_{A, m}\) & \(p_{B, m}\) \\
\midrule()
\endhead
1 & 0.6019 & 0.4001 \\
101 & 0.3172 & 0.2766 \\
201 & 0.3375 & 0.2805 \\
301 & 0.3812 & 0.2503 \\
401 & 0.2590 & 0.3362 \\
\bottomrule()
\end{longtable}

Thus, firm \(A\) enters 165 markets while firm \(B\) enters 179.
However, recall that Bayesian Nash equilibrium allows for ex-post regret
where profits are negative succeeding the decision of entry. In our
case, firm \(A\) regrets 41 times while \(B\) regrets 48---24.85\% and
26.82\%, respectively.

\newpage

\hypertarget{maximum-likelihood}{%
\section{Maximum Likelihood}\label{maximum-likelihood}}

In pursuit of our estimation objectives for
\(\theta = (\alpha, \beta, \gamma)\) such that \(\gamma\) appears in
\(\nu_m := \gamma \times \mathbf{1} \{ m \leq 100 \}\)---i.e., we also
want to estimate a common market fixed effect for the first 100
markets---we define our log-likelihood function as shown below.

\begin{align*}
\log\mathcal{L}_{m} (\theta \mid \cdot) &= \sum_{m = 1}^{M} \left[ D_{A, m} D_{B, m} \log(p_{A, m} p_{B, m}) \right. \\
&+ D_{A, m} (1 - D_{B, m}) \log(p_{A, m} (1 - p_{B, m})) \\
&+ (1 - D_{A, m}) D_{B, m} \log((1 - p_{A, m}) p_{B, m}) \\
&+ \left. (1 - D_{A, m}) (1 - D_{B, m}) \log((1 - p_{A, m}) (1 - p_{B, m})) \right]
\end{align*}

The maximum likelihood estimates, as well as their standard errors, are
shown in the following table.

\begin{longtable}[]{@{}ccc@{}}
\toprule()
Parameter & Estimate & SE \\
\midrule()
\endhead
\(\alpha\) & 2.0347 & 0.1813 \\
\(\beta\) & 0.1915 & 0.0470 \\
\(\gamma\) & 0.7421 & 0.1865 \\
\bottomrule()
\end{longtable}

\newpage

\hypertarget{gmm}{%
\section{GMM}\label{gmm}}

With the generalized method of moments (GMM), we may want to consider up
to the fourth moment to ensure over-identification. This way, we can
describe an algorithm that performs GMM estimation as follows:

\begin{enumerate}
\def\labelenumi{\arabic{enumi}.}
\tightlist
\item
  Compute
  \(\theta^{(0)} = \arg\min_\theta \bar{g}(\theta)'\bar{g}(\theta)\)
\item
  Compute the heteroskedasticity and auto-correlation consistent matrix
  \(\hat{\Omega} \left(\theta^{(0)}\right)\) like the one proposed by
  \citet{HAC}
\item
  Compute
  \(\theta^{(0)} = \arg\min_\theta \bar{g}(\theta)'\left[\hat{\Omega} \left(\theta^{(0)}\right)\right]^{-1}\bar{g}(\theta)\)
\item
  If
  \(\left\lVert \theta^{(0)} - \theta^{(1)} \right\rVert < \text{tol}\),
  stop. Else, \(\theta^{(0)} = \theta^{(1)}\) and repeat from 2
\item
  Define \(\hat{\theta} = \theta^{(1)}\)
\end{enumerate}

\newpage

\bibliography{ref.bib}


\end{document}
