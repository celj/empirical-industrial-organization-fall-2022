% see http://info.semprag.org/basics for a full description of this template
\documentclass[cm,linguex]{glossa}

% possible options:
% [times] for Times font (default if no option is chosen)
% [cm] for Computer Modern font
% [lucida] for Lucida font (not freely available)
% [brill] open type font, freely downloadable for non-commercial use from http://www.brill.com/about/brill-fonts; requires xetex
% [charis] for CharisSIL font, freely downloadable from http://software.sil.org/charis/
% for the Brill an CharisSIL fonts, you have to use the XeLatex typesetting engine (not pdfLatex)
% for headings, tables, captions, etc., Fira Sans is used: https://www.fontsquirrel.com/fonts/fira-sans
% [biblatex] for using biblatex (the default is natbib, do not load the natbib package in this file, it is loaded automatically via the document class glossa.cls)
% [linguex] loads the linguex example package
% !! a note on the use of linguex: in glossed examples, the third line of the example (the translation) needs to be prefixed with \glt. This is to allow a first line with the name of the language and the source of the example. See example (2) in the text for an illustration.
% !! a note on the use of bibtex: for PhD dissertations to typeset correctly in the references list, the Address field needs to contain the city (for US cities in the format "Santa Cruz, CA")

%\addbibresource{sample.bib}
% the above line is for use with biblatex
% replace this by the name of your bib-file (extension .bib is required)
% comment out if you use natbib/bibtex

\let\B\relax %to resolve a conflict in the definition of these commands between xyling and xunicode (the latter called by fontspec, called by charis)
\let\T\relax
\usepackage{xyling} %for trees; the use of xyling with the CharisSIL font produces poor results in the branches. This problem does not arise with the packages qtree or forest.
\usepackage[linguistics]{forest} %for nice trees!
\usepackage{longtable}

\title[Empirical Industrial Organization]{Final Exam}
% Optional short title inside square brackets, for the running headers.

% \author[Paul \& Vanden Wyngaerd]% short form of the author names for the running header. If no short author is given, no authors print in the headers.
% {%as many authors as you like, each separated by \AND.
%   \spauthor{Waltraud Paul\\
%   \institute{CNRS, CRLAO}\\
%   \small{105, Bd. Raspail, 75005 Paris\\
%   waltraud.paul@ehess.fr}
%   }
%   \AND
%   \spauthor{Guido Vanden Wyngaerd \\
%   \institute{KU Leuven}\\
%   \small{Warmoesberg 26, 1000 Brussel\\
%   guido.vandenwyngaerd@kuleuven.be}
%   }%
% }

\author[Carlos Lezama]{
    \spauthor{Carlos Lezama\\
  \institute{\hfill\break
Instituto Tecnológico\\
Autónomo de México}\\
  \small{\hfill\break
clezamaj@itam.mx}
  }%
  }

\usepackage{natbib}


% tightlist command for lists without linebreak
\providecommand{\tightlist}{%
  \setlength{\itemsep}{0pt}\setlength{\parskip}{0pt}}





\usepackage{multirow, array}
\usepackage{amsmath, amsthm}

\setlength{\parindent}{0pt}
\setlength{\parskip}{\baselineskip}

\newcommand{\dnorm}{\mathcal{N}}
\newcommand{\dunif}{\mathcal{U}}
\newcommand{\ev}{\text{E}}
\newcommand{\iid}{\text{i.i.d.}}

\newtheoremstyle{defn}
{}                % Space above
{}                % Space below
{\mdseries}        % Theorem body font % (default is "\upshape")
{}                % Indent amount
{\bfseries\sffamily}       % Theorem head font % (default is \mdseries)
{.}               % Punctuation after theorem head % default: no punctuation
{ }               % Space after theorem head
{}                % Theorem head spec
\theoremstyle{defn}
\newtheorem{defn}{Definition}

\newtheoremstyle{axiom}
{}                % Space above
{}                % Space below
{\mdseries}        % Theorem body font % (default is "\upshape")
{}                % Indent amount
{\bfseries\sffamily}       % Theorem head font % (default is \mdseries)
{.}               % Punctuation after theorem head % default: no punctuation
{ }               % Space after theorem head
{}                % Theorem head spec
\theoremstyle{axiom}
\newtheorem{axiom}{A\!\!}

\newtheoremstyle{thm}
{}                % Space above
{}                % Space below
{\mdseries}        % Theorem body font % (default is "\upshape")
{}                % Indent amount
{\bfseries\sffamily}       % Theorem head font % (default is \mdseries)
{.}               % Punctuation after theorem head % default: no punctuation
{ }               % Space after theorem head
{}                % Theorem head spec
\theoremstyle{thm}
\newtheorem{thm}{Theorem}

\newtheoremstyle{lem}
{}                % Space above
{}                % Space below
{\mdseries}        % Theorem body font % (default is "\upshape")
{}                % Indent amount
{\bfseries\sffamily}       % Theorem head font % (default is \mdseries)
{.}               % Punctuation after theorem head % default: no punctuation
{ }               % Space after theorem head
{}                % Theorem head spec
\theoremstyle{lem}
\newtheorem{lem}[thm]{Lemma}

\newtheoremstyle{cor}
{}                % Space above
{}                % Space below
{\mdseries}        % Theorem body font % (default is "\upshape")
{}                % Indent amount
{\bfseries\sffamily}       % Theorem head font % (default is \mdseries)
{.}               % Punctuation after theorem head % default: no punctuation
{ }               % Space after theorem head
{}                % Theorem head spec
\theoremstyle{cor}
\newtheorem{cor}{Corollary}[thm]

\newtheoremstyle{prop}
{}                % Space above
{}                % Space below
{\mdseries}        % Theorem body font % (default is "\upshape")
{}                % Indent amount
{\bfseries\sffamily}       % Theorem head font % (default is \mdseries)
{.}               % Punctuation after theorem head % default: no punctuation
{ }               % Space after theorem head
{}                % Theorem head spec
\theoremstyle{prop}
\newtheorem{prop}{Proposition}

\newtheoremstyle{rmk}
{}                % Space above
{}                % Space below
{\itshape}        % Theorem body font % (default is "\upshape")
{}                % Indent amount
{\slshape\sffamily}       % Theorem head font % (default is \mdseries)
{.}               % Punctuation after theorem head % default: no punctuation
{ }               % Space after theorem head
{}                % Theorem head spec
\theoremstyle{rmk}
\newtheorem*{rmk}{Remark}

\begin{document}


\sffamily
\maketitle



\rmfamily

%  Body of the article
\maketitle
\thispagestyle{empty}

In this exam we investigate a static continuous game with complete
information in the supply side. We consider two scenarios: one where the
firms have constant marginal costs, and another one where firms have
increasing marginal costs. We study the solution and estimation of this
game.

\hypertarget{constant-marginal-costs}{%
\section{Constant Marginal Costs}\label{constant-marginal-costs}}

Suppose that we have already estimated a demand model, thus we have
estimates for the demand function of the firm, \(q_j(p)\). Suppose, too,
that the profit function of the firm is

\[ \Pi_j(p) = p_jq_j(p) - c_j\left(q_j(p)\right) , \]

where \(c_j\left(q_j(p)\right) = c_0 + c_1q_j(p) + Z_j\gamma + \nu_j\),
in which \(c_0\), \(c_1\), and \(\gamma\) are unknown (to the
econometrician) parameters to be estimated, \(Z_j\) is a vector of cost
shifters, and \(\nu_j\) is a mean zero unobserved (to the
econometrician) determinant of costs.

Note that the marginal cost function is

\[ c_j'(q_j) = c_1 q_j'(p). \]

This way, our first order condition, \(\Pi_j'(p) = 0\), yields to

\[ q_j'(p) (p_j - c_1) = 0 \iff p_j = c_1 \]

Using the estimated demand function, \(\hat q_j(p)\), the expression for
the firm's first order condition, and assuming no endogeneity concerns,
we way follow \citet{hackmann} to estimate
\(\theta = (c_0, c_1, \gamma)\) through GMM. Under this suitable
conditions this estimator is consistent, asymptotically normal, and,
potentially, also asymptotically efficient.\footnote{With right choice
  of a positive-definite weighting matrix.}

Notice that we would need to adjust our standard errors, because we are
using an estimate of \(q_j(p)\), which induces measurement error in your
estimation. To fix this, we may easily apply a two-stage estimation such
2SLS. The advantage of 2SLS estimators over other IV estimators is that
2SLS can easily combine multiple instrumental variables, and it also
makes including control variables easier.

However, while easy to implement, the main drawback of two-stage models
has been that the estimation of standard errors from the second stage
alone are incorrect because they ignore the measurement error that
carries over from using the predictions of one model in the next model.

According to \citet{quasi}, the model described before can also be fit
using quasi-- limited-information maximum likelihood. As described in
the article, for cases in which the models are linear (2SLS), and under
the assumption that the errors of the models distribute normal and
independent, the joint maximum log-likelihood function can be written as

\[ L(\theta_1, \theta_2) = L_1(\theta_1) + L_2(\theta_2, \hat{\theta}_1). \]

This way, the estimation of this model would not require additional
adjustments on the estimation of the standard errors, because the
measurement error from \(\theta_1\) is already accounted for in the
model.

\hypertarget{increasing-marginal-costs}{%
\section{Increasing Marginal Costs}\label{increasing-marginal-costs}}

Now, suppose that the cost function is quadratic, and of the form

\[ c_j\left(q_j(p)\right) = c_0 + c_1q_j(p) + c_2q_j(p)^2 + Z_j\gamma + \nu_j , \]

where \(c_0\), \(c_1\), \(c_2\), and \(\gamma\) are unknown (to the
econometrician) parameters to be estimated.

Note that the marginal cost function is

\[ c_j'(q_j) = c_1 q_j'(p) + 2c_2q_j(p)q_j'(p). \]

This way, our first order condition, \(\Pi_j'(p) = 0\), yields to

\[ q_j'(p) (p_j - c_1 - 2c_2q_j(p)) = 0 \iff p_j = c_1 + 2c_2q_j(p) \]

Using the estimated demand function, \(\hat q_j(p)\), the expression for
the firm's first order condition, and assuming no endogeneity concerns,
we may easily apply maximum likelihood estimation on \(\theta\) with

\[ \ell(p; \theta) = \sum_j \log(\Pi_j(p)) \]

\bibliography{ref.bib}


\end{document}
